\documentclass[twoside,11pt]{article}
/home/arya/workspace/projects/latex/macros-arya.tex
% Any additional packages needed should be included after jmlr2e.
% Note that jmlr2e.sty includes epsfig, amssymb, natbib and graphicx,
% and defines many common macros, such as 'proof' and 'example'.
%
% It also sets the bibliographystyle to plainnat; for more information on
% natbib citation styles, see the natbib documentation, a copy of which
% is archived at http://www.jmlr.org/format/natbib.pdf

\usepackage{jmlr2e}

% Definitions of handy macros can go here

%\newcommand{\dataset}{{\cal D}}
%\newcommand{\fracpartial}[2]{\frac{\partial #1}{\partial  #2}}

% Heading arguments are {volume}{year}{pages}{submitted}{published}{author-full-names}

\jmlrheading{}{}{}{}{}{}

% Short headings should be running head and authors last names

\ShortHeadings{}{}
\firstpageno{1}

\usepackage[linesnumbered,ruled]{algorithm2e}
\begin{document}

\title{DataRank: An Online Ranking Algorithm for Ranking Biomedical Datasets}

\author{\name Arya Iranmehr, MS \email airanmehr@ucsd.edu \\
       \addr Department of Electrical and Computer Engineering\\
       \AND
       \name Huan Wang, MS \email huanwng@ucsd.edu \\
       \addr Department of Computer Science and Engineering\\
       \AND
       \name Xiaoqian Jiang, PhD \email x1jiang@ucsd.edu \\
       \addr Division of Biomedical Informatics\\
       University of California, San Diego\\
       La Jolla, CA 92037, USA
       }

\editor{}

\maketitle
\begin{abstract} 
In this paper, we propose an online ranking algorithm, DataRank for ranking biomedical datasets that are used in the papers index in PubMed Central. DataRank's input is a bipartite citation graph between datasets and articles which each paper is represented by set of corresponding MeSH terms. DataRank works by imputing a set of MeSH terms to each dataset as features, by aggregating MeSH terms from the connected papers in the bipartite graph. For each search query, DataRank first maps the query to set of MeSH terms and present a \emph{offline} ranking of datasets for the MeSH-Query using a Bayesian approach which the likelihood is proportional to Jaccard index and prior is proportional to number of citations of that dataset. DataRank is also extended to a \emph{online} algorithm by incorporating user-feedbacks regarding ranking relevance. The online DataRank again takes an Bayesian approach  which uses offline DataRank as its prior and computes its likelihood by estimating the user rating for unknown values using collaborative filtering. A demo web search engine has been developed to rank more than 20,000 dataset that has been discovered in more than 1 million papers.
\end{abstract} 


\section{Introduction} \label{sec:introduction}
\cite{ptm} 
\section{Background} \label{sec:background}
\section{Methodology} \label{sec:methodology}
\section{Implementation} \label{sec:implementation}
\section{Experiments} \label{sec:Experiments}
\section{Conclusions} \label{sec:conclusions}

\bibliography{/home/arya/Documents/library}

\end{document}
